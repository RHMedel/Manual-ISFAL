\chapter*{Prefacio}
\addcontentsline{toc}{chapter}{Prefacio}

Este texto es producto de un ejercicio de trabajo colaborativo utilizando herramientas digitales que son usadas regularmente en comunidades de software libre\footnote{Herramientas utilizadas: el sistema de control de versiones Git, la plataforma de repositorios GitHub y el procesador de textos \LaTeX.}, realizado entre los años 2016 y 2018. Si bien su ambicioso título es \emph{Manual de la asignatura Ingeniería de Software de Fuentes Abiertas/Libre}, más bien debe considerarse como una serie de apuntes sobre los diferentes temas abordados en esta asignatura electiva (de quinto nivel) de la carrera de Ingeniería en Sistemas de Infomación de la Facultad Regional Córdoba de la Universidad Tecnológica Nacional.

Estos apuntes, entonces, no deben tomarse como el texto definitivo de la asignatura, ya que ninguno de los capítulos cubre todos los aspectos de cada tema ni profundiza en ellos. Más bien, se recomienda utilizarlo como una base para comenzar una exploración más profunda de cada tema, a partir de las referencias y enlaces provistos.

Como todo trabajo colaborativo, este texto es la suma de los aportes de cada autor/a y la interacción entre autores/as, por lo que se puede notar cierta heterogeneidad en los estilos de escritura y los enfoques con los que se aborda cada tema. Hemos hecho ciertos intentos de homogeneización del estilo del texto, pero aún queda trabajo por hacer en ese sentido.

Durante los años en que se escribió y modificó este texto, muchas han sido las personas que hicieron sus aportes, no todas dejaron su registro\footnote{Si considerás que sos autor/a y no se te incluyó en esta lista, por favor, enviame un email a \texttt{rmedel@frc.utn.edu.ar} indicando que participaste en la creación de este manual.}, por lo que esta lista puede estar incompleta: 
Pablo Bajo,
Luciano Bartoszensky,
Federico Benito,
Agustín Borello,
Braian Chincho,
Nahuel Ignacio Cuello,
Alexis Donato,
Manolo Fernández, 
Facundo Ferrero,
Juan Filardo,
Máximo Fiora, 
Mayco Garelli,   
Marcos López,
Carlos Luna,
Paolo Mattio,
Ricardo Medel,
Fernando Meichtri,
Diego Moreno Moreyra, 
Jeremías Niño,
Federico Prado,
Fernanda Pucheta, 
Julieta Ríos,
Mauricio Sánchez,
Lucas Martín Segurado,
Fabricio Simoncelli,
Milena Vilardo,
Gonzalo Ulla,
Fernando Zamora.

A todas ellas y todos ellos, va el agradecimiento de la cátedra, por su aporte a este texto que nos permite tener un punto de partida en común para poder desarrollar cada uno de los temas que abordamos en esta asignatura única en su tipo, al menos por ahora, en todo el país.
 
Si las lectoras o los lectores encuentran errores o quieren aportar mejoras, son libres de hacerlo por su propia iniciativa, ya que este documento tiene licencia libre y su código fuente está accesible en un repositorio público\footnote{\url{https://github.com/RHMedel/Manual-ISFAL}}.

\parbox{2in}{\vspace{2in}}
\hfill
\parbox{2in}%
{Ricardo Medel\\%
Córdoba, 8 de marzo de 2022\\%
\mbox{}} 

\noindent Este trabajo está licenciado bajo la Creative Commons Attribution-ShareAlike 4.0 International License (\url{http://creativecommons.org/licenses/by-sa/4.0/}).

