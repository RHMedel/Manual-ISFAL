\chapter{La filosofía del software libre en otros ámbitos}
%\chapterauthor{Bajo, Moreno}

Tanto el software libre como su filosofía de trabajo comunitario, de compartir libremente la información y de aportes voluntarios han tenido impacto en otros ámbitos por fuera del desarrollo de software.

\section{Producción de bienes artísticos}

Existe un movimiento dedicado a la creación artística colaborativa, con obras que puedan ser compartidas y modificadas por cualquier persona. Estos/as creadores/as de obras de arte, en particular las digitales, ya sean gráficos, libros, documentos, imágenes, etc., quieren asegurar que sus obras se compartan y utilicen en distintos ámbitos, pero que no sean apropiadas por empresas para generar ganancias sin compartirlas con la comunidad. Para esto se han creado diferentes licencias que aseguran el respeto por los derechos de autor mientras permiten la generación de obras derivadas. 

De particular éxito en el ámbito de obras artísticas ha sido la familia de licencias de Creative Commons\footnote{Creative Commons: \url{https://creativecommons.org/}}, la cual es una organización sin fines de lucro dedicada a promover el acceso y el intercambio de cultura. Ha desarrollado un conjunto de instrumentos jurídicos de carácter gratuito que facilitan usar y compartir tanto la creatividad como el conocimiento.

Con estos instrumentos el/la autor/a de una obra puede, de una manera simple y estandarizada, otorgar permisos para compartir y usar su trabajo creativo bajo los términos y condiciones de su elección. Las licencias Creative Commons están compuestas por cuatro módulos de condiciones, los cuales al ser seleccionados permiten otrogar diferentes tipos de permisos a los/as usuarios/as:

\begin{itemize}
\item Atribución o \emph{Attribution} (BY): requiere que en cualquier explotación de la obra será necesario reconocer al/la autor/a explícitamente.

\item No comercial o \emph{Non Commercial} (NC): la explotación de la obra queda limitada a usos no comerciales. Es decir, no se puede utilizar la obra para ganar dinero.

\item Sin obras derivadas o \emph{No Derivate Works} (ND): la autorización para explotar la obra no incluye la posibilidad de crear una obra derivada. Se puede utilizar y comercializar (si así lo permite la licencia) pero solamente la obra tal como fue creada por su autor/a, sin modificaciones.

\item Compartir igual o \emph{Share Alike} (SA): la explotación autorizada incluye la creación de obras derivadas siempre que mantengan la misma licencia al ser distribuidas.
\end{itemize}

Con estas condiciones se pueden generar distintas combinaciones que producen las licencias Creative Commons:

\begin{itemize}
\item Atribución (BY): Se permite cualquier explotación de la obra, incluyendo la explotación con fines comerciales y la creación de obras derivadas, la distribución de las cuales también está permitida sin ninguna restricción. Esta licencia es una licencia libre.

\item Reconocimiento – Compartir Igual (BY-SA): Se permite el uso comercial de la obra y de las posibles obras derivadas, la distribución de las cuales se debe hacer con una licencia igual a la que regula la obra original. Esta licencia es una licencia libre.

\item Atribución – No Comercial (BY-NC): Se permite la generación de obras derivadas siempre que no se haga con fines comerciales y citando al/la autor/a original. Tampoco se puede utilizar la obra original con fines comerciales. Esta licencia no es una licencia libre.

\item Atribución – No Comercial – Compartir Igual (BY-NC-SA): No se permite un uso comercial de la obra original ni de las posibles obras derivadas, la distribución de las cuales se debe hacer con una licencia igual a la que regula la obra original. Esta licencia no es una licencia libre.

\item Atribución – Sin Obras Derivadas (BY-ND): Se permite el uso comercial de la obra, citando al/la autor/a, pero no la generación de obras derivadas. Esta licencia no es una licencia libre.

\item Atribución – No Comercial – Sin Obras Derivadas (BY-NC-ND): No se permite un uso comercial de la obra original ni la generación de obras derivadas. Esta licencia no es una licencia libre, y es la más cercana al derecho de autor tradicional.

\end{itemize}


Otras licencias libres también pueden utilizarse para libros, en particular para la documentación de los proyectos de software:

\begin{itemize}

\item {\bf Licencia de documentación libre de GNU o GFDL}: es una licencia libre robusta para documentos diseñada por la FSF GNU. Permite asegurar que el material esté disponible de forma completamente libre, pudiendo ser copiado, redistribuido, modificado e incluso vendido siempre y cuando el material se mantenga bajo los términos de esta misma licencia. 

\item {\bf FreeBSD Documentation License}: es una licencia de documentación libre permisiva, creada para su uso con el sistema operativo FreeBSD, incompatible con la GFDL.

\item {\bf Apple's Common Documentation License}: es una licencia de documentación libre, incompatible con la GFDL, puesto que la sección (2c) indica ``No agregue otros términos ni condiciones a esta licencia'' y, además, la GFDL posee cláusulas adicionales que no están contempladas en la Common Documentation License.

\item {\bf Open Publication License}: es una licencia libre robusta que se utiliza para documentación, siempre y cuando el titular del copyright no ejerza ninguna de las ``OPCIONES DE LA LICENCIA'' que se mencionan en la sección VI. Pero si se invoca cualquiera de esas opciones, la licencia se vuelve privativa. En cualquier caso, es incompatible con la GFDL.
    
\end{itemize}


\section{Software libre en el Estado}

El software libre impacta de varias maneras en las administraciones públicas: permite un mejor aprovechamiento de recursos, fomenta la industria local, ofrece independencia de proveedores monopólicos, permite el escrutinio público y garantiza el acceso a los datos a largo plazo.

\begin{itemize}
\item  \textbf{Aprovechamiento más adecuado de los recursos:} Muchas aplicaciones utilizadas o promovidas por las administraciones públicas son también utilizadas por muchos otros sectores de la sociedad. Por ello, cualquier inversión pública en el desarrollo de un producto libre redundará en beneficios no sólo en la propia administración, sino en todos los ciudadanos u otras administraciones (municipales, provinciales, nacionales) que puedan encontrarle utilidad.

\item  \textbf{Fomento de la industria local:} Una de las mayores ventajas del software libre es la posibilidad de desarrollar industria local de software. Cuando se usa software propietario, todo lo gastado en licencias va directamente al fabricante del producto, y además esa compra redunda en el fortalecimiento de su posición. Lo cual no es necesariamente perjudicial, pero poco eficiente para las regiones vinculadas a las diferentes administraciones, comparado con la alternativa de usar un programa libre. En este caso, las empresas locales podrán competir proporcionando servicios (y el propio programa) a la administración, en condiciones muy similares a cualquier otra empresa.

\item \textbf{Independencia de proveedor:} Cualquier organización preferirá depender de un mercado en régimen de competencia que de un solo proveedor que puede imponer las condiciones en que proporciona su producto. Sin embargo, en el mundo de la administración esta preferencia se convierte en requisito fundamental, y hasta en obligación legal en algunos casos. La administración no puede, en general, elegir contratar a un proveedor cualquiera, sino que debe especificar sus necesidades de forma que cualquier empresa interesada, que cumpla unas ciertas características técnicas, y que proporcione el servicio o el producto demandado con una cierta calidad, pueda optar a un contrato.

En el caso del software propietario, para cada producto no hay más que un proveedor. Si se especifica un producto dado, se está decidiendo también qué proveedor contratará con la administración. Y en muchos casos es prácticamente imposible evitar especificar un cierto producto cuando estamos hablando de programas de ordenador. Razones de compatibilidad dentro de la organización, o de ahorros en formación y administración, hacen habitual que una administración decida usar un cierto producto.

La única salida a esta situación es que el producto especificado sea libre. En ese caso, cualquier empresa interesada podrá proporcionarlo, y también cualquier tipo de servicio sobre él. Además, en caso de contratar de esta manera, la administración pertinente podrá en el futuro cambiar inmediatamente de proveedor si así lo desea, y sin ningún problema técnico, pues aunque cambie de empresa, el producto que usará será el mismo.

\item \textbf{Adaptación a las necesidades exactas:} Aunque la adaptación a sus necesidades exactas es algo que necesita cualquier organización que precisa de la informática, las peculiaridades de la administración hacen que éste sea un factor muy importante para el éxito de la implantación de un sistema informático. En el caso de usar software libre, la adaptación puede hacerse con mucha mayor facilidad, y lo que es más importante, sirviéndose de un mercado con competencia, si hace falta contratarla.

Cuando una administración compra un producto propietario, modificarlo pasa normalmente por alcanzar un acuerdo con su productor, que es el único que legalmente (y muchas veces técnicamente) puede hacerlo. En esas condiciones, es difícil realizar buenas negociaciones, sobre todo si el productor no está  excesivamente interesado en el mercado que le ofrece la administración en cuestión. Sin embargo, usando un producto libre, esa administración puede modificarlo a su antojo, si dispone de personal para ello, o contratar externamente la modificación. Esta contratación la puede realizar en principio cualquier empresa que tenga los conocimientos y capacidades para ello, por lo que es de esperar que haya concurrencia de varias empresas. Este hecho, necesariamente tiende a abaratar los costes y a mejorar la calidad.

\item \textbf{Escrutinio público de seguridad:} Para una administración pública poder garantizar que sus sistemas informáticos hacen sólo lo que está previsto que hagan es un requisito fundamental, y en muchos estados, un requisito legal. No son pocas las veces que esos sistemas manejan datos privados, en los que pueden estar interesados muchos terceros (pensemos en datos fiscales, penales, censales, electorales, etc.). Difícilmente, si se usa una aplicación propietaria sin código fuente disponible, puede asegurarse que efectivamente esa aplicación trata esos datos como debe. Pero incluso si se ofrece su código fuente, las posibilidades que tendrá una institución pública para asegurar que no contiene código extraño serán muy limitadas. Sólo si se puede encargar ese trabajo de forma habitual y rutinaria a terceros, y cualquier parte interesada puede escrutarlos, podrá estar la administración razonablemente segura de cumplir con ese deber fundamental, o al menos de tomar todas las medidas en su mano para hacerlo.

\item \textbf{Disponibilidad a largo plazo:} Muchos datos que manejan las administraciones y los programas
que sirven para calcularlos han de estar disponibles por decenas de años o incluso siglos. Es muy difícil asegurar que un programa propietario cualquiera estará disponible cuando hayan pasado esos periodos de tiempo, y más si lo que se desea es que funcione en la plataforma habitual en ese momento futuro. Por el contrario, es muy posible que su productor haya perdido interés en el producto y no lo haya portado a nuevas plataformas, o que sólo esté dispuesto a hacerlo ante grandes contraprestaciones económicas. De nuevo, hay que recordar que sólo él puede hacer ese porte, y por lo tanto será difícil negociar con él. En el caso del software libre, por el contrario, la aplicación está disponible con seguridad para que cualquiera la porte y la deje funcionando según las necesidades de la administración. Si eso no sucede de forma espontánea, la administración siempre puede dirigirse a varias empresas buscando la mejor oferta para hacer el trabajo. De esta forma puede garantizarse que la aplicación y los datos que maneja
estarán disponibles cuando haga falta.

\end{itemize}


Si bien las ventajas de uso del software libre en las administraciones son muchas, también son muchas las dificultades cuando se enfrentan a su puesta en práctica.

\begin{itemize}

\item \textbf{Desconocimiento y falta de decisión política:} El primer problema que se encuentra el software libre para su introducción en las administraciones es uno que comparte con otras organizaciones: es aún muy desconocido entre quienes toman las decisiones. Éste es un problema que cada vez se va solucionando mejor, pero en muchos ámbitos de las administraciones el software libre aún es percibido como algo extraño, sin garantías o hecho por amateurs, y tomar la decisión de usarlo implica asumir ciertos riesgos técnicos y/o políticos.

\item \textbf{Poca adecuación de los mecanismos de contratación:} Los mecanismos de contratación que se usan hoy día en la administración, desde los modelos de concurso público habituales hasta la división del gasto en partidas, están diseñados fundamentalmente para la compra de productos informáticos, y no tanto para la adquisición de servicios relacionados con los programas. Sin embargo, cuando se utiliza software libre, habitualmente no hay producto que comprar, o su precio es casi despreciable. Por el contrario, para aprovechar las posibilidades que ofrece el software libre, es conveniente poder contratar servicios a su alrededor. Esto hace necesario que, antes de utilizar seriamente software libre, se hayan diseñado mecanismos burocráticos adecuados que faciliten la contratación en estos casos.

\item \textbf{Falta de estrategia de implantación:} En muchos casos, el software libre comienza a usarse en una administración simplemente porque el coste de adquisición es más bajo. Es muy habitual en estos casos que el producto en cuestión se incorpore al sistema informático sin mayor planificación, y en general sin una estrategia global de uso y aprovechamiento de software libre. Esto causa que la mayor parte de las ventajas del software libre se pierdan en el camino, ya que todo queda en el uso de un producto más barato.

Si a esto unimos que la migración de un software privativo de uso extendido en la administración a un software libre supone costes de transición no despreciables, hace que experiencias aisladas de uso de software libre en la administración puedan resultar fallidas y frustrantes.

\item \textbf{Escasez o ausencia de productos libres en ciertos segmentos:} La intención de implantar software libre en cualquier organización puede chocar con la falta de alternativas libres de la calidad adecuada para cierto tipo de aplicaciones. En esos casos, la solución es complicada o riesgosa: tratar de promover la aparición del producto libre que se necesita. Las administraciones públicas están en una buena posición para estudiar seriamente si les conviene fomentar, o incluso financiar o cofinanciar el desarrollo de ese producto.

\end{itemize}

Las administraciones públicas actúan sobre el mundo del software al menos de tres maneras:

\begin{itemize}

\item Comprando programas y servicios relacionados con ellas. Las administraciones, como grandes usuarios de informática, son un actor fundamental en el mercado del software.

\item Promoviendo de diversas formas el uso (y la adquisición) de ciertos programas en la sociedad. Esta promoción se hace a veces ofreciendo incentivos económicos (desgravaciones fiscales, incentivos directos, etc.), a veces con información y recomendaciones.

\item Financiando (directa o indirectamente) proyectos de investigación y desarrollo que están diseñando el futuro de la informática.

\end{itemize}

\subsection{Seguridad de los datos de los ciudadanos}

El Estado debe almacenar y procesar información relativa a los ciudadanos. La relación entre el individuo y el Estado depende de la privacidad e integridad de estos datos, que por consiguiente deben ser adecuadamente resguardados contra tres riesgos específicos:
 
\begin{itemize}
\item  Riesgo de filtración: los datos confidenciales deben ser tratados de tal manera que el acceso a ellos sea posible exclusivamente para las personas e instituciones autorizadas.

\item Riesgo de imposibilidad de acceso: los datos deben ser almacenados de tal forma que el acceso a ellos por parte de las personas e instituciones autorizadas esté garantizado durante toda la vida útil de la información.

\item Riesgo de manipulación: la modificación de los datos debe estar restringida, nuevamente, a las personas e instituciones autorizadas.
\end{itemize}

La concreción de cualquiera de estas tres amenazas puede tener consecuencias graves tanto para el Estado como para el individuo. El software libre permite al usuario la inspección completa y exhaustiva del mecanismo mediante el cual procesa los datos. Sin la posibilidad de la inspección, es imposible saber si el programa cumple con su función, o si además incluye vulnerabilidades intencionales o accidentales que permitan a terceros acceder indebidamente a los datos, o impedir que los usuarios legítimos de la información puedan usarlo. El hecho de permitir la inspección del programa es una excelente medida de seguridad, ya que al estar expuestos los mecanismos, estos están constantemente a la vista de profesionales capacitados, con lo que se vuelve inmensamente más difícil ocultar funciones maliciosas, aún si el usuario final no se toma el trabajo de buscarlas él mismo.

%{\bf Software Libre que utiliza el Gobierno}
%\begin{itemize}
%\item  OpenStack: Sistema capaz de realizar diversas funciones en la nube, automatización, nubes privadas y publicas, etc, infinidad de cosas que el sistema OpenStack permite realizar y que por supuesto fue nacida y creada precisamente por el gobierno.
%\item Jenkins: Herramienta que es capaz de encontrar todo tipo de errores en el código fuente de un sistema. Es por eso que las empresas de gobierno utilizan tanto Jenkins. El objetivo es tener una serie de sistemas libres de errores, nada de fugas, nada de entradas externas, con Jenkins todo queda blindado y solucionado. 
%\item WineHQ: Esta aplicación, lo que hace es que es capaz de emular el sistema operativo de Windows. De esta forma no importa que sistema operativo utilices, podrás hacer uso de tus programas tradicionales sin ningún problema. Te recuerdo que WineHQ también es libre, por lo que la podrás descargar fácilmente de internet.
%\end{itemize}

\subsection{Leyes de uso de software en el Estado}

Con lo antes dicho hemos visto que el software tiene un profundo impacto en las actividades realizadas por el Estado. Los riesgos que involucra una elección desafortunada no son menores: imposibilidad de auditar la función pública, falta de garantías por parte del Estado sobre la manipulación de la información privada de sus ciudadanos, dependencia de un proveedor para el correcto desempeño de las funciones del Estado, imposibilidad de los ciudadanos de acceder a su información, entre otras. 

Es claro entonces que la adquisición de software por parte del Estado debiera estar regulada por una ley que, previendo este tipo de situaciones, fije las condiciones bajo las cuales el proveedor debe suministrar los programas en cuestión. No puede dejarse librada a cada funcionario responsable de un área de la administración pública la decisión de las condiciones de contratación o compra de software, dado los peligros que podría acarrear para el conjunto de la comunidad una elección desafortunada.

%Santa Fe
%Ecuador?
%Venezuela
%Argentina (rechazada)

\subsection{Voto Electrónico}

Una discusión que recurrentemente se presenta en el discurso público es el voto electrónico como mejora al sistema actual de elecciones de autoridades~\cite{lanacion17, serra17}. Lo incluimos en esta sección por dos motivos: el licenciamiento libre ha sido presentado como una característica necesaria del software de voto electrónico y las votaciones para elegir gobiernos son, aunque no las únicas elecciones de las que los ciudadanos participamos (piénsese en un club o un colegio profesional), son, sin duda, las más importantes para nuestra vida social.

La primera consideración sobre el voto electrónico es que debe cumplir los requerimientos de toda elección democrática: el voto debe ser secreto, debe registrarse la intención del/la votante, solo los y las electoras autorizadas pueden votar y cada elector/a debe emitir solamente un único voto (incluyendo la opción de votar en blanco). 

Un requisito usual es que cualquier ciudadano/a pueda auditar el proceso. Esto hace improbable que cualquier código privativo, con su habitual calidad de secreto comercial, pueda ser utilizado en una elección transparente. Por eso muchas veces se apela a la necesidad de que el software utilizado para las votaciones sea de código abierto.

En el mundo solo unos pocos países o regiones utilizan voto electrónico, debido a la dificultad de demostrar que cumplen con los requisitos indicados anteriormente de forma clara y transparente~\cite{smaldone16}. Los países utilizando voto electrónico son Venezuela, Brasil e India. Mientras que se utiliza en algunas jurisdicciones de Estados Unidos, Perú, Argentina, Francia, Japón y Somalía. En nuestro país se utiliza en Salta, Ciudad Autónoma de Buenos Aires y parcialmente Córdoba~\cite{origlia16} para elecciones locales, pero su uso a nivel nacional fue descartado luego del debate en el Senado de la fallida ley de reforma electoral impulsado por el Gobierno Nacional, que tenía ya la media sanción de la Cámara de Diputados~\cite{ybarra16}

Respecto de países donde se dejó de utilizar el voto electrónico, en Alemania la Corte Suprema declaró en 2009 a los sistemas de voto electrónico como inconstitucionales (por que no podían ser auditados por un ciudadano común), en Holanda dejaron de utilizarse en 2007 al comprobarse que los votos podían ser leídos a varios metros de distancia (en algunas máquinas) y que los programas podían ser alterados, mientras que en Irlanda se evaluó un sistema en elecciones piloto (entre 2002 y 2004, con un costo estimado de 54 millones de euros) y se determinó que no se podía garantizar la integridad de ninguna elección que usara ese sistema, que fue dado de baja definitivamente en 2009.

Dos interesantes fuentes de información sobre voto electrónico, desde la posición opuesta a él, son los dos libros editados por la Fundación Vía Libre\footnote{Fundación Vía Libre: \url{https://www.vialibre.org.ar}}: ``Voto electrónico: los riesgos de una ilusión''~\cite{busaniche08} y ``Voto electrónico: una solución en busca de problemas''~\cite{busaniche17}.

\section{Software libre en la educación}

Analizaremos los distintos aspectos relacionados con el uso de software en la educación de ciudadanos/as. Desarrollaremos dicho análisis desde tres enfoques diferentes: el económico, el académico y el ético.

\begin{itemize}

\item {\bf Aspecto económico}

Aunque, como ya indicamos repetidamente, la libertad del software no implica su gratuidad, esto no invalida el hecho de que el software propietario es altamente costoso tanto en su adquisición como en su mantenimiento y que el acceso al software libre es, en principio, gratuito.

El alto precio de las licencias de uso, en las que se basa el modelo de comercialización y distribución propietario, hacen que los montos que debieran invertirse en la adquisición del derecho de utilización sean muy elevados. Dichos costos contrastan notoriamente con los costos de adquisición de los productos licenciados como software libre.
 
Cabe resaltar una diferencia radical entre ambos modelos: en el caso del software propietario, lo que se adquiere al pagar el precio de una licencia es el permiso de ejecutar el programa en cuestión bajo determinadas condiciones, en tanto que al adquirir una pieza de software libre se obtiene una copia del programa (incluyendo su código fuente) y el permiso del autor para hacer uso de las cuatro libertades que definen a este tipo de software.

El impacto de las licencias del software propietario no sólo debe tenerse en cuenta a la hora de evaluar los costos de equipar un laboratorio informático en una institución educativa, sino que también debe atenderse al hecho de que aquellos alumnos que quieran utilizar las mismas herramientas en su hogar también deberán adquirir dichas licencias. 

\item {\bf Aspecto académico}

Aunque el objetivo de la educación no es formar especialistas en determinadas tecnologías, es indudable que a la hora de aplicar los conceptos teóricos debe optarse por alguna en particular. Es una obligación, por parte del docente, el seleccionar cuidadosamente las tecnologías más con venientes a tal efecto.

En el caso del software, a través de la historia, una serie de empresas -que han gozado de una situación de monopolio- han pretendido marcar el rumbo tecnológico. Invariablemente, estas empresas han tratado de ejercer presión sobre el sistema académico para lograr el acercamiento de los estudiantes a sus productos, invirtiendo en muchos casos elevadas sumas de dinero a tal efecto (bajo la forma de convenios o donaciones). 

Creemos que el uso de herramientas propietarias, generalmente ligadas a tecnologías específicas bajo el control de determinadas empresas, contribuye a lograr una fuerte dependencia de los y las estudiantes, en cuanto ciudadanos digitales. Cada uno/a de ellos/as actuará como un agente de ventas de dicha empresa al utilizar, recomendar y hasta exigir soluciones basadas exclusivamente en los productos que conoce.

Por contrapartida, al utilizar herramientas libres, se le brinda al estudiante la posibilidad de conocer los detalles de la implementación de las mismas y los fundamentos tecnológicos en los que éstas se basan. 

\item {\bf Aspecto ético}

Con relación al alto costo de las licencias de uso, es una actividad muy común en  la actualidad la  utilización de ``copias no autorizadas'' de productos propietarios, bajo la permisiva mirada de los y las docentes. La práctica de este tipo de actividades conlleva las siguientes consecuencias:

\begin{itemize}
    \item La violación de las condiciones de la licencia correspondiente  al producto utilizado de forma irregular constituye lisa y llanamente una violación de la ley.

    \item El uso de copias no autorizadas contribuye a la estrategia utilizada por las empresas productoras de software propietario para promocionar sus productos y provocando así el llamado ``efecto de red'', ya que al tratarse de tecnología cerrada y bajo el control de dichas empresas, se está propiciando el establecimiento de la misma como un estándar ``de facto''.

\end{itemize}
    
Los y las docentes están, entonces, promoviendo o al menos permitiendo, que se viole la ley, lo cual está en las antípodas del comportamiento ético que se espera de ellos/as.

\end{itemize}

\subsection{Necesidad de una política institucional}

Ante los peligros que puede acarrear la incorrecta elección de herramientas propietarias en la educación, creemos que ésta es una decisión que no puede quedar librada al criterio de cada docente, sino que debe definirse una política institucional que determine los lineamientos para la elección de herramientas y tecnologías informáticas en las distintas actividades desarrolladas dentro de su ámbito.

Un ejemplo de software libre utilizado por una decisión institucional en nuestro país es el sistema operativo Huayra\footnote{Huayra: \url{https://huayra.educar.gob.ar/}}, el cual es una distribución GNU/Linux desarrollada por el Estado Nacional. Se incorporó desde 2013 en las netbooks que el Programa Conectar Igualdad (PCI\footnote{Programa Conectar Igualdad: \url{https://conectarigualdad.edu.ar/inicio}}) entrega a estudiantes secundarios de escuelas públicas de todo el país. El PCI, aún antes de incorporar a Huayra, incluía en las netbooks software libre para uso educativo, incluso en el sistema operativo privativo con el que vienen todas las máquinas en modo \emph{doble booteo}\footnote{Con excepción de las entregadas en 2021 bajo el Plan Federal Juana Manso, que solo incluyen Huayra versión 5~\cite{distefano21}}.

\subsection{Recursos Educativos Abiertos}

Una derivación de la filosofía del software libre en la educación es la generación y uso de Recursos Educativos Abiertos (REA), esto es, material educativo creado por investigadores/as, pedagogos/as, docentes y estudiantes que se distribuye con licencias libres, lo cual permite utilizarlo en cualquier institución educativa sin tener que pagar, pero también modificarlo para adaptarlo a las necesidades de cada curso en particular. 

En Argentina, tanto el portal Educ.ar\footnote{Educ.ar: \url{https://www.educ.ar/}} como el del Programa Conectar Igualdad proveen materiales didácticos, muchos de los cuales son abiertos y desarrollados en las diferentes jurisdicciones educativas del país y compartidos con el resto del sistema educativo. En el caso del PCI todos los contenidos están bajo la licencia Creative Commons Reconocimiento-NoComercial-CompartirIgual 4.0 Internacional (CC BY-NC-SA 4.0)\footnote{Licencia CC BY-NC-SA 4.0: \url{https://creativecommons.org/licenses/by-nc-sa/4.0/deed.es_ES}}, excepto cuando se declare lo contrario\footnote{Condiciones de uso de los contenidos del sitio Conectar Igualdad, artículo 6.1. USO DE LOS CONTENIDOS: \url{https://conectarigualdad.edu.ar/condiciones}}.


\section{Software libre en organizaciones sociales}

Las organizaciones y movimientos sociales tienen mucho en común, dados sus principios filosóficos, jurídicos, sociales y económicos, con el movimiento del software libre. De esta manera el software libre puede ser incorporado como herramienta tecnológica en estas organizaciones sin traicionar a sus principios.

\begin{itemize}

\item {\bf Principios filosóficos}

Ambos consideran a la libertad y la solidaridad como principios éticos fundamentales, y colocan al beneficio de la humanidad por sobre todo otro interés. En particular, consieran al conocimiento como un bien público que beneficia a la colectividad en general y permite el desarrollo igualitario, por lo que el software, en tanto conocimiento codificado, debe difundirse sin trabas.

\item {\bf Razones jurídicas}

Evita los problemas de uso ilegal de licencias de software privativo y sus posibles implicaciones legales (demandas, multas, entre otros).

\item {\bf Razones socioeconómicas}

Estos movimientos promueven la economía solidaria y el comercio justo, especialmente la que se desarrolla en términos cooperativos y participativos, considerando a las personas usuarias como sujetos de derechos, no solo consumidores de productos.

Usualmente estas organizaciones no cuentan con muchos recursos económicos y se necesita destinar los que tiene para el logro de su misión y objetivos centrales, por lo que las licencias de software libre permiten acceder a tecnología que de otra forma les estaría vedada.

Finalmente, el software libre es amigable con el ambiente, una preocupación central de estos movimientos sociales, al no requerir de la sustitución tan frecuente de los equipos, reduciendo así la producción de basura tecnológica.

\item {\bf Accesibilidad y seguridad:}

La accesibilidad, otro tema recurrente en estos movimientos, puede ser garantizada al permitir el software libre su adaptación a personas, contextos e idiomas.

El software libre mejora la seguridad al permitir la corrección de fallas y problemas de forma oportuna  y según las necesidades de los/as usuarios/as. En particular, mucha de la información que manejan las organizaciones de la sociedad civil suele ser considerada ``confidencial'' y necesita ser resguardada. El software libre permite asegurar que la información no es compartida con otras instituciones o personas.

\end{itemize}


